% Options for packages loaded elsewhere
\PassOptionsToPackage{unicode}{hyperref}
\PassOptionsToPackage{hyphens}{url}
%
\documentclass[
  letterpaper,
  10pt,
  openany]{book}

\usepackage{amsmath,amssymb}
\usepackage{iftex}
\ifPDFTeX
  \usepackage[T1]{fontenc}
  \usepackage[utf8]{inputenc}
  \usepackage{textcomp} % provide euro and other symbols
\else % if luatex or xetex
  \usepackage{unicode-math}
  \defaultfontfeatures{Scale=MatchLowercase}
  \defaultfontfeatures[\rmfamily]{Ligatures=TeX,Scale=1}
\fi
\usepackage{lmodern}
\ifPDFTeX\else  
    % xetex/luatex font selection
\fi
% Use upquote if available, for straight quotes in verbatim environments
\IfFileExists{upquote.sty}{\usepackage{upquote}}{}
\IfFileExists{microtype.sty}{% use microtype if available
  \usepackage[]{microtype}
  \UseMicrotypeSet[protrusion]{basicmath} % disable protrusion for tt fonts
}{}
\makeatletter
\@ifundefined{KOMAClassName}{% if non-KOMA class
  \IfFileExists{parskip.sty}{%
    \usepackage{parskip}
  }{% else
    \setlength{\parindent}{0pt}
    \setlength{\parskip}{6pt plus 2pt minus 1pt}}
}{% if KOMA class
  \KOMAoptions{parskip=half}}
\makeatother
\usepackage{xcolor}
\setlength{\emergencystretch}{3em} % prevent overfull lines
\setcounter{secnumdepth}{5}


\providecommand{\tightlist}{%
  \setlength{\itemsep}{0pt}\setlength{\parskip}{0pt}}\usepackage{longtable,booktabs,array}
\usepackage{calc} % for calculating minipage widths
% Correct order of tables after \paragraph or \subparagraph
\usepackage{etoolbox}
\makeatletter
\patchcmd\longtable{\par}{\if@noskipsec\mbox{}\fi\par}{}{}
\makeatother
% Allow footnotes in longtable head/foot
\IfFileExists{footnotehyper.sty}{\usepackage{footnotehyper}}{\usepackage{footnote}}
\makesavenoteenv{longtable}
\usepackage{graphicx}
\makeatletter
\def\maxwidth{\ifdim\Gin@nat@width>\linewidth\linewidth\else\Gin@nat@width\fi}
\def\maxheight{\ifdim\Gin@nat@height>\textheight\textheight\else\Gin@nat@height\fi}
\makeatother
% Scale images if necessary, so that they will not overflow the page
% margins by default, and it is still possible to overwrite the defaults
% using explicit options in \includegraphics[width, height, ...]{}
\setkeys{Gin}{width=\maxwidth,height=\maxheight,keepaspectratio}
% Set default figure placement to htbp
\makeatletter
\def\fps@figure{htbp}
\makeatother

% Begin ims-style.tex ----------------------------------------------------------

% Output formats ---------------------------------------------------------------

% PDF

\usepackage[bookmarksnumbered, pdfborder = {0 0 0}, urlcolor = oiGB, colorlinks = true, linkcolor = oiGB, citecolor = oiGB, backref = true]{hyperref}

% Packages ---------------------------------------------------------------------

\usepackage[framemethod=tikz]{mdframed} 
\usepackage[normalem]{ulem}
%\usepackage{subfigure}
\usepackage[explicit]{titlesec}
\usepackage{
  amsmath, calc,
  %footnote, 
  fancyhdr,
  geometry, graphicx,
  makeidx,
  booktabs,
  longtable,
  array,
  multirow,
  wrapfig,
  float,
  colortbl,
  pdflscape,
  tabu,
  threeparttable,
  threeparttablex,
  makecell,
  xcolor,
  caption,
  subcaption
}
%\newcommand{\UnderscoreCommands}{%
%  \do\citep \do\citeP
%}
\usepackage[bottom]{footmisc}
\usepackage[strings]{underscore} % must be last

% Colors  on screen ------------------------------------------------------------

%\definecolor{oiB}{HTML}{569BBD}            % COL["blue","full"]
%\definecolor{oiLB}{HTML}{e3eef4}           % lighter version of oiB
%
%\definecolor{oiY}{HTML}{f4dc00}            % COL["yellow","full"]
%\definecolor{oiLY}{HTML}{fffacd}           % lighter version of oiY
%
%\definecolor{oiR}{HTML}{E97583}            % COL["red","full"]
%\definecolor{oiLR}{HTML}{F3CED4}           % lighter version of oiR
%
%\definecolor{oiGray}{HTML}{808080}         % COL["gray","full"]
%\definecolor{oiLGray}{HTML}{f8f8f8}        % lighter version of oiGray
%
%\definecolor{oiGB}{rgb}{0.5,0.5,.5}        % from OS4 - for footnotes

% Colors - for print -----------------------------------------------------------

\definecolor{oiB}{HTML}{000000}            % COL["blue","full"]
\definecolor{oiLB}{HTML}{e0e0e0}           % lighter version of oiB

\definecolor{oiY}{HTML}{000000}            % COL["yellow","full"]
\definecolor{oiLY}{HTML}{e0e0e0}           % lighter version of oiY

\definecolor{oiR}{HTML}{000000}            % COL["red","full"]
\definecolor{oiLR}{HTML}{e0e0e0}           % lighter version of oiR

\definecolor{oiGray}{HTML}{808080}         % COL["gray","full"]
\definecolor{oiLGray}{HTML}{f8f8f8}        % lighter version of oiR

\definecolor{oiGB}{rgb}{0.5,0.5,.5}        % from OS4 - for footnotes

% Headers 

\fancypagestyle{plain}{%
\fancyhf{} % clear all header and footer fields
\fancyhead[RO,RE]{\thepage} %RO=right odd, RE=right even
\renewcommand{\headrulewidth}{0pt}
\renewcommand{\footrulewidth}{0pt}
\renewcommand{\footruleskip}{5pt}}
\raggedbottom

% Make index

\makeindex

% add space between table rows 

\setlength{\defaultaddspace}{10pt}

% footnotes - try to stop them from running accross two pages

\interfootnotelinepenalty=10000

%-------------------------------------------------------------------------------
% From OS4 style.tex
%
% 1 Page Parameters
% 1.1
\setlength\paperheight{11in}
\setlength\paperwidth{8.5in}
\newcommand{\officialtextheight}{9.7in}
\newcommand{\officialtextwidth}{6.1in}
%\setlength\paperheight{10in}
%\setlength\paperwidth{8in}
%\newcommand{\officialtextheight}{8.7in}
%\newcommand{\officialtextwidth}{6in}
\newcommand{\officialvoffset}{-0.6in}
\setlength\textheight{\officialtextheight}
\setlength\textwidth{\officialtextwidth}
\setlength\voffset{\officialvoffset}
\renewcommand{\baselinestretch}{1.0}

% 1.2 Margin Size
\setlength\hoffset{0.25in}
% 1.2.1 Even
\setlength\oddsidemargin{0in}
\setlength\evensidemargin{0in}
% 1.2.2 Slightly offset
\setlength\oddsidemargin{0.08in}
\setlength\evensidemargin{-0.08in}
% 1.2.3 Significant offset
% WARNING: The chapter pages will show partially hidden page numbers.
%\setlength\oddsidemargin{0.2in}
%\setlength\evensidemargin{-0.2in}
% 1.3 PDF Parameters
%\setlength\paperheight{11in}
%\setlength\textheight{8.25in}
%\setlength\paperwidth{8.5in}
%\setlength\textwidth{5.45in}
%\setlength\voffset{-10mm}
%\setlength\oddsidemargin{0.75in}
%\setlength\evensidemargin{0.75in}
% 1.4 Margin Spacing
\setlength{\marginparsep}{5mm}
\setlength{\marginparwidth}{20mm}

% 1.5 Page Header

\renewcommand{\headrulewidth}{0pt}
\fancyhead[RO,LE]{\thepage}
\fancyhead[RE]{\leftmark}
\fancyhead[LO]{\rightmark}
\fancyfoot[c]{}
\fancyheadoffset[RO,LE]{0.9in}

% 6.6 Hyperreferences
\newcommand{\oiRedirect}[2]{\href{http://www.openintro.org/redirect.php?go=#1&referrer=\referrer}{#2}}

% Fonts

\usepackage[utf8]{inputenc}
\usepackage[scaled]{helvet}
\renewcommand{\familydefault}{\sfdefault}
\newcommand{\Helvetica}{\sffamily}

% Chapter and section titles

\titleformat{\chapter}[display]
{\color{oiB}\normalfont\Huge\bfseries\Helvetica}
{\color{oiB}Chapter \thechapter}{1em}{#1}

\titleformat{\section}
{\color{oiB}\normalfont\Large\bfseries\Helvetica}
{\color{oiB}\thesection}{1em}{#1}

\titleformat{\subsection}
{\color{oiB}\normalfont\large\bfseries\Helvetica}
{\color{oiB}\thesubsection}{1em}{#1}

\titleformat{\subsubsection}
{\color{oiB}\normalfont\normalsize\bfseries\Helvetica}
{\color{oiB}\thesubsubsection}{1em}{#1}

% IMS specific style -----------------------------------------------------------

% Helper environments

% mdframedwithfootChapterintro: for chapterintro box
% currently none of them increment footnote counter,
% this seems to be the right approach?

\newenvironment{mdframedwithfootChapterintro}
{
    \savenotes
    \begin{mdframed}[%
    topline=true, bottomline=true, linecolor=oiB, linewidth=1.4pt,
    rightline=false, leftline=false,
    backgroundcolor=oiLB]
    %\stepcounter{footnote} % don't increment footnote counter
    \renewcommand{\thempfootnote}{\arabic{footnote}}
    }
{
    \end{mdframed}
    \spewnotes
}

% mdframedwithfootGPWE: for guidedpractice and workedexample

\newenvironment{mdframedwithfootGPWE}
{
    \savenotes
    \begin{mdframed}[%
    topline=true, bottomline=true, linecolor=oiB, linewidth=0.5pt,
    rightline=false, leftline=false,
    backgroundcolor=oiLGray]
    %\stepcounter{footnote}
    \renewcommand{\thempfootnote}{\arabic{footnote}}
    }
{
    \end{mdframed}
    \spewnotes
}

% mdframedwithfootImportant: for important

\newenvironment{mdframedwithfootImportant}
{
    \savenotes
    \begin{mdframed}[%
    topline=true, bottomline=true, linecolor=oiR, linewidth=0.5pt,
    rightline=false, leftline=false,
    backgroundcolor=oiLGray]
    %\stepcounter{footnote}
    \renewcommand{\thempfootnote}{\arabic{footnote}}
    }
{
    \end{mdframed}
    \spewnotes
}

% mdframedwithfootTip: for tip, data, and pronunciation

\newenvironment{mdframedwithfootTipDataPro}
{
    \savenotes
    \begin{mdframed}[%
    topline=true, bottomline=true, linecolor=oiGray, linewidth=0.5pt,
    rightline=false, leftline=false,
    backgroundcolor=oiLGray]
    %\stepcounter{footnote}
    \renewcommand{\thempfootnote}{\arabic{footnote}}
    }
{
    \end{mdframed}
    \spewnotes
}

% mdframedwithfootTutLab: for tutorials and labs

\newenvironment{mdframedwithfootTutLab}
{
    \savenotes
    \begin{mdframed}[%
    topline=false, bottomline=false,
    rightline=false, leftline=false]
    \stepcounter{footnote}
    \renewcommand{\thempfootnote}{\arabic{footnote}}
    }
{
    \end{mdframed}
    \spewnotes
}


% Custom environments/boxes -------------------------------------------------------
% based on imsstyle.css

% uses regular mdframedwithfoot variants, might have footnotes

% chapterintro

\newenvironment{chapterintro}{
\vspace{4mm}
\begin{mdframedwithfootChapterintro}
\begin{minipage}[t]{0.10\textwidth}
{$\:$ \\ \setkeys{Gin}{width=2.5em,keepaspectratio}\includegraphics{images/_icons/chapterintro.png}}
\end{minipage}
\hfill
\begin{minipage}[t]{0.90\textwidth}
\setlength{\parskip}{1em}
\large
}{\end{minipage}
\end{mdframedwithfootChapterintro}
\vspace{4mm}
}

% guidedpractice

\newenvironment{guidedpractice}{
\vspace{4mm}
\begin{mdframedwithfootGPWE}
\begin{minipage}[t]{0.10\textwidth}
{$\:$ \\ \setkeys{Gin}{width=2.5em,keepaspectratio}\includegraphics{images/_icons/guided-practice.png}}
\end{minipage}
\hfill
\begin{minipage}[t]{0.90\textwidth}
%\vspace{-2mm}
\setlength{\parskip}{1em}
\noindent\textbf{\color{oiB}\small\Helvetica{\MakeUppercase{Guided Practice}}} $\:$ \\ 
}{\end{minipage}
\end{mdframedwithfootGPWE}
\vspace{4mm}
}

% workedexample

\newenvironment{workedexample}{
    \let\oldrule\rule
    \renewcommand{\rule}[2]{\vspace{-2mm}\oldrule{##1}{##2}\vspace{-2mm}}
\vspace{4mm}
\begin{mdframedwithfootGPWE}
\begin{minipage}[t]{0.10\textwidth}
{$\:$ \\ \setkeys{Gin}{width=2.5em,keepaspectratio}\includegraphics{images/_icons/worked-example.png}}
\end{minipage}
\hfill
\begin{minipage}[t]{0.90\textwidth}
%\vspace{-2mm}
\setlength{\parskip}{1em}
\noindent\textbf{\color{oiB}\small\Helvetica{\MakeUppercase{Example}}} $\:$ \\
\small}{\end{minipage}
\end{mdframedwithfootGPWE}
\vspace{4mm}
}

% exercises

\newenvironment{exercises}{
    \let\oldtextbf\textbf
    \renewcommand{\textbf}[1]{{\textcolor{oiB}{\oldtextbf{##1}}}}
    \renewcommand{\labelenumi}{\thechapter.\arabic{enumi}.}
}

% important

\newenvironment{important}{
    \let\oldtextbf\textbf
    \renewcommand{\textbf}[1]{{\textcolor{oiR}{\oldtextbf{##1}}}}
\vspace{4mm}
\begin{mdframedwithfootImportant}
\begin{minipage}[t]{0.10\textwidth}
{$\:$ \\ \setkeys{Gin}{width=2.5em,keepaspectratio}\includegraphics{images/_icons/important.png}}
\end{minipage}
\hfill
\begin{minipage}[t]{0.90\textwidth}
%\vspace{-2mm}
\setlength{\parskip}{1em}
}{\end{minipage}
\end{mdframedwithfootImportant}
\vspace{4mm}
}

% tip

\newenvironment{tip}{
\vspace{4mm}
\begin{mdframedwithfootTipDataPro}
\begin{minipage}[t]{0.10\textwidth}
{$\:$ \\ \setkeys{Gin}{width=2em,keepaspectratio}\includegraphics{images/_icons/tip.png}}
\end{minipage}
\hfill
\begin{minipage}[t]{0.90\textwidth}
%\vspace{-2mm}
\setlength{\parskip}{1em}
}{\end{minipage}
\end{mdframedwithfootTipDataPro}
\vspace{2mm}
}

% data

\newenvironment{data}{
\vspace{4mm}
\begin{mdframedwithfootTipDataPro}
\begin{minipage}{0.10\textwidth}
{\setkeys{Gin}{width=2em,keepaspectratio}\includegraphics{images/_icons/data.png}}
\end{minipage}
\hfill
\begin{minipage}{0.90\textwidth}
%\vspace{-2mm}
\setlength{\parskip}{1em}
}{\end{minipage}
%\vspace{-2mm}
\end{mdframedwithfootTipDataPro}
\vspace{2mm}
}

% pronunciation

\newenvironment{pronunciation}{
\vspace{2mm}
\begin{mdframedwithfootTipDataPro}
\begin{minipage}{0.10\textwidth}
{$\:$ \\ \setkeys{Gin}{width=2em,keepaspectratio}\includegraphics{images/_icons/pronunciation.png}}
\end{minipage}
\hfill
\begin{minipage}{0.90\textwidth}
%\vspace{-2mm}
\setlength{\parskip}{1em}
}{\end{minipage}
\end{mdframedwithfootTipDataPro}
\vspace{2mm}
}


% uses regular mdframed, won't have footnotes

% singletutorial

\newenvironment{singletutorial}{
%\vspace{2mm}
\begin{mdframed}[topline=false, bottomline=false, rightline=false, leftline=false]
\begin{minipage}{0.05\textwidth}
$\:$ \\
\end{minipage}
\begin{minipage}{0.10\textwidth}
{$\:$ \\ \setkeys{Gin}{width=2em,keepaspectratio}\includegraphics{images/_icons/singletutorial.png}}
\end{minipage}
\hfill
\begin{minipage}{0.85\textwidth}
%\vspace{-2mm}
\setlength{\parskip}{1em}
}{\end{minipage}
\end{mdframed}
%\vspace{1mm}
}

% alltutorials

\newenvironment{alltutorials}{
\vspace{2mm}
\begin{mdframed}[topline=false, bottomline=false, rightline=false, leftline=false]
\begin{minipage}{0.10\textwidth}
{$\:$ \\ \setkeys{Gin}{width=2em,keepaspectratio}\includegraphics{images/_icons/alltutorials.png}}
\end{minipage}
\hfill
\begin{minipage}{0.90\textwidth}
%\vspace{-2mm}
\setlength{\parskip}{1em}
}{\end{minipage}
\end{mdframed}
%\vspace{2mm}
}

% singlelab

\newenvironment{singlelab}{
\vspace{2mm}
\begin{mdframed}[topline=false, bottomline=false, rightline=false, leftline=false]
\begin{minipage}{0.10\textwidth}
{$\:$ \\ \setkeys{Gin}{width=2em,keepaspectratio}\includegraphics{images/_icons/singlelab.png}}
\end{minipage}
\hfill
\begin{minipage}{0.90\textwidth}
%\vspace{-2mm}
\setlength{\parskip}{1em}
}{\end{minipage}
\end{mdframed}
\vspace{2mm}
}

% todo

\newenvironment{todo}{
\vspace{4mm}
\begin{mdframed}[%
    topline=true, bottomline=true, linecolor=oiY, linewidth=0.5pt,
    rightline=false, leftline=false,
    backgroundcolor=oiLY]
\begin{minipage}[t]{0.10\textwidth}
{$\:$ \\ \setkeys{Gin}{width=2em,keepaspectratio}\includegraphics{images/_icons/to-do.png}}
\end{minipage}
\hfill
\begin{minipage}[t]{0.90\textwidth}
\vspace{-2mm}
\setlength{\parskip}{1em}
\noindent\textbf{\color{oiGray}\small\Helvetica{{\MakeUppercase{TO DO}}}} $\:$ \\ \\
}{\end{minipage}
\end{mdframed}
\vspace{4mm}
}

% underconstruction

\newenvironment{underconstruction}{
\vspace{4mm}
\begin{mdframed}[%
    topline=true, bottomline=true, linecolor=oiR, linewidth=0.5pt,
    rightline=false, leftline=false,
    backgroundcolor=oiLR]
\begin{minipage}[t]{0.10\textwidth}
{$\:$ \\ \setkeys{Gin}{width=2em,keepaspectratio}\includegraphics{images/_icons/under-construction.png}}
\end{minipage}
\hfill
\begin{minipage}[t]{0.90\textwidth}
\vspace{-2mm}
\setlength{\parskip}{1em}
\noindent\textbf{\color{oiR}\small\Helvetica{{\MakeUppercase{Under construction}}}} $\:$ \\ \\
}{\end{minipage}
\end{mdframed}
\vspace{4mm}
}

% Cover image ------------------------------------------------------------------

\newenvironment{authorinfo}[1]
  {
  \begin{minipage}[c]{0.30\textwidth}
  {\setkeys{Gin}{width=12em,keepaspectratio}\includegraphics{#1}}
  \end{minipage} 
  \hfill
  \begin{minipage}[c]{0.60\textwidth}
  }
  {
  \end{minipage}
  }

% Part formatting --------------------------------------------------------------

\titleformat{\part}[display]
{\color{oiR}\titlerule[5pt]\vspace{3pt}\color{oiLR}\titlerule[2pt]\vspace{3pt}\color{oiB}\normalfont\Huge\bfseries\scshape\Helvetica}
{\color{oiB}PART \thepart}{2em}{#1 \\ \noindent \vspace{3pt}\color{oiR}\titlerule[5pt]}

% https://tex.stackexchange.com/questions/506428/background-color-for-section-title
%\titleformat{\part}{\LARGE}{\rlap{\color{oiLB}\rule[-0.4cm]{\linewidth}{5cm}}\color{oiB}\normalfont\Huge\bfseries\scshape\Helvetica PART \thepart}{1em}{\color{oiB}\normalfont\Huge\bfseries\scshape\Helvetica #1}


% Bibliography: Chapter should be called References ----------------------------

%\usepackage{natbib}
%\usepackage{bibentry}
%\makeatletter\let\saved@bibitem\@bibitem\makeatother
%\usepackage{hyperref}
%\makeatletter\let\@bibitem\saved@bibitem\makeatother

% Index

\usepackage{makeidx}
\makeindex

% Smaller captions with bold labels

\usepackage[font=small,labelfont=bf]{caption}

% Reduce space between chapters in TOC

\usepackage{tocbasic}
\DeclareTOCStyleEntry[
  beforeskip=.5em plus 1pt,% default is 1em plus 1pt
  pagenumberformat=\textbf
]{tocline}{chapter}

% End ims-style.tex ------------------------------------------------------------
\makeatletter
\@ifpackageloaded{bookmark}{}{\usepackage{bookmark}}
\makeatother
\makeatletter
\@ifpackageloaded{caption}{}{\usepackage{caption}}
\AtBeginDocument{%
\ifdefined\contentsname
  \renewcommand*\contentsname{Table of contents}
\else
  \newcommand\contentsname{Table of contents}
\fi
\ifdefined\listfigurename
  \renewcommand*\listfigurename{List of Figures}
\else
  \newcommand\listfigurename{List of Figures}
\fi
\ifdefined\listtablename
  \renewcommand*\listtablename{List of Tables}
\else
  \newcommand\listtablename{List of Tables}
\fi
\ifdefined\figurename
  \renewcommand*\figurename{Figure}
\else
  \newcommand\figurename{Figure}
\fi
\ifdefined\tablename
  \renewcommand*\tablename{Table}
\else
  \newcommand\tablename{Table}
\fi
}
\@ifpackageloaded{float}{}{\usepackage{float}}
\floatstyle{ruled}
\@ifundefined{c@chapter}{\newfloat{codelisting}{h}{lop}}{\newfloat{codelisting}{h}{lop}[chapter]}
\floatname{codelisting}{Listing}
\newcommand*\listoflistings{\listof{codelisting}{List of Listings}}
\makeatother
\makeatletter
\makeatother
\makeatletter
\@ifpackageloaded{caption}{}{\usepackage{caption}}
\@ifpackageloaded{subcaption}{}{\usepackage{subcaption}}
\makeatother

\ifLuaTeX
  \usepackage{selnolig}  % disable illegal ligatures
\fi
\usepackage{bookmark}

\IfFileExists{xurl.sty}{\usepackage{xurl}}{} % add URL line breaks if available
\urlstyle{same} % disable monospaced font for URLs
\hypersetup{
  pdftitle={International Master of Environmental Sciences},
  pdfauthor={IMES Examination Committee},
  hidelinks,
  pdfcreator={LaTeX via pandoc}}


\title{International Master of Environmental Sciences}
\usepackage{etoolbox}
\makeatletter
\providecommand{\subtitle}[1]{% add subtitle to \maketitle
  \apptocmd{\@title}{\par {\large #1 \par}}{}{}
}
\makeatother
\subtitle{Module Handbook}
\author{IMES Examination Committee}
\date{2025-04-04}

\begin{document}
\frontmatter
\maketitle

\renewcommand*\contentsname{Table of contents}
{
\setcounter{tocdepth}{0}
\tableofcontents
}

\mainmatter
\bookmarksetup{startatroot}

\chapter*{Preface}\label{preface}
\addcontentsline{toc}{chapter}{Preface}

\markboth{Preface}{Preface}

This is a module handbook for the Master Program International Master of
Environmental Sciences at the University of Cologne. The document is
work in progress.

\bookmarksetup{startatroot}

\chapter*{Contact}\label{contact}
\addcontentsline{toc}{chapter}{Contact}

\markboth{Contact}{Contact}

\textbf{Chair of the Examination Committee}

Prof.~Dr.~Christina Bogner

Ecosystem Research

\textbf{IMES Academic Manager}

Dr.~Hannes Laermanns

imes-info(at)uni-koeln.de

\bookmarksetup{startatroot}

\chapter*{Some usefull concepts and abbreviations}\label{sec-concepts}
\addcontentsline{toc}{chapter}{Some usefull concepts and abbreviations}

\markboth{Some usefull concepts and abbreviations}{Some usefull concepts
and abbreviations}

\textbf{ECTS}: The European Credit Transfer and Accumulation System
(ECTS) is a standardized system used across the European Higher
Education Area (EHEA) to facilitate the recognition of academic
achievements and support student mobility between institutions. ECTS
credits represent the workload and defined learning outcomes of a given
course or program. One ECTS credit at UoC corresponds to approximately
30 hours of work, including lectures, self study, and assessments. A
full academic year typically equals 60 ECTS credits, approximately 30
per semester, making the system a key tool for ensuring the
transparency, comparability, and quality assurance of higher education
qualifications in Europe.

\textbf{Module}: a thematic teaching unit, usually consisting of several
courses.

\textbf{Contact time}: time the student spends in class.

\textbf{Self-study time}: time the students dedicate to any work done at
home like assignments, studying to rework concepts learnt in class or
prepare an exam.

\textbf{Non-graded assessment (German Studienleistung)}: mandatory
formative assessment at UoC, which must be completed satisfactorily for
progression but does not contribute to the final grade of a course. Such
coursework might involve participation, practical work, or assignments
designed to demonstrate understanding or competence without being graded
on a scale that affects overall degree outcomes.

\textbf{Core Modules (CM)}: These modules provide foundational knowledge
in essential areas of environmental sciences. They introduce key
concepts, methodologies, and frameworks and are typically completed
early in the program.

\textbf{Advanced Modules (AM)}: Building on the Core Modules, these
modules allow students to deepen their knowledge, refine their skills,
and engage with more complex environmental issues. They encourage
independent application of methods and further development of analytical
and research competencies.

\textbf{Specialisation Modules (SM)}: These modules enable students to
focus on a specific area of interest within environmental sciences.
Specialisation Modules support the development of an individual academic
profile and typically correspond with the topic of the Master's thesis.

\textbf{SoSe}: Summer semester or summer term from 1 April to 30
September.

\textbf{WiSe}: Winter semester or winter term from 1 October to 31
March.

\part{DESCRIPTION OF THE PROGRAM}

\chapter*{The International Master of Environmental Sciences
(IMES)}\label{the-international-master-of-environmental-sciences-imes}
\addcontentsline{toc}{chapter}{The International Master of Environmental
Sciences (IMES)}

\markboth{The International Master of Environmental Sciences (IMES)}{The
International Master of Environmental Sciences (IMES)}

\section*{Purpose}\label{purpose}
\addcontentsline{toc}{section}{Purpose}

\markright{Purpose}

Environmental challenges today are multifaceted, systemic, and global in
scope. They cannot be addressed by isolated disciplines, isolated
geographies or short-term solutions. The International Master of
Environmental Sciences (IMES) program responds to this complexity by
fostering a learning environment rooted in interdisciplinarity,
internationality, scientific inquiry, and anticipatory thinking.

Inspired by the Club of Rome's vision of
\href{https://www.clubofrome.org/blog-post/bologna-qa-legacy-of-no-limits-to-learning/}{No
Limits to Learning}, the program views education not merely as knowledge
transmission, but as a transformative and continuous process. Students
are encouraged to integrate perspectives from the natural sciences,
social sciences, humanities, law, and economics to better understand and
engage with human--environment systems. Through this process, they
develop the capacity to learn across boundaries --- disciplinary,
cultural, and institutional --- and to respond with creativity and
responsibility to emerging environmental challenges.

Aligned with the values articulated in
\href{https://thefifthelement.earth/}{The Fifth Element}, the IMES
program promotes a systems-based approach to education. It seeks to
cultivate not only analytical and methodological competence, but also a
sense of responsibility, relational thinking, and openness to
regeneration --- both ecological and societal. The program cultivates
environmental stewardship as a dynamic, reflective practice rooted in
scientific insight, ethical awareness, and cross-cultural dialogue. In
this way, IMES prepares students to co-create knowledge and solutions in
a world that demands profound and continuous learning.

\section*{Quick facts}\label{quick-facts}
\addcontentsline{toc}{section}{Quick facts}

\markright{Quick facts}

\begin{longtable}[]{@{}ll@{}}
\toprule\noalign{}
\endhead
\bottomrule\noalign{}
\endlastfoot
\textbf{Degree} & Master of Science (M.Sc.) \\
\textbf{Duration} & 4 semesters / 2 years \\
\textbf{Credits} & 120 ECTS \\
\textbf{Language} & English \\
\textbf{Application Deadline} & May 15 (annually) \\
\end{longtable}

\section*{Study Objectives}\label{study-objectives}
\addcontentsline{toc}{section}{Study Objectives}

\markright{Study Objectives}

The IMES program (Master of Science) prepares students to engage in
independent, research-based inquiry into environmental challenges at the
interface of natural and human systems. Students learn to critically
evaluate scientific findings, apply appropriate methodologies, and
analyze complex environmental problems from interdisciplinary and
international perspectives.

The program fosters the ability to integrate knowledge across
environmental sciences, including ecological, legal, economic, and
societal dimensions. Students strengthen their scientific writing,
presentation, and project management skills while developing the social
competencies required to work successfully in transdisciplinary
contexts. Through a combination of theoretical depth and applied
experience, the IMES program equips graduates for diverse career paths
in environmental science and policy, and for academic advancement at the
doctoral level.

\section*{Content}\label{content}
\addcontentsline{toc}{section}{Content}

\markright{Content}

IMES is a four-semester, interdisciplinary degree program that combines
natural and social sciences to investigate pressing environmental
challenges. Core modules introduce key scientific, legal, economic, and
social dimensions of human--environment interactions. A set of
integrative modules connects these domains, supporting a systems-level
understanding of environmental issues.

The program fosters an international learning environment through a
diverse cohort of students from EU and non-EU countries, and through the
participation of guest lecturers from partner institutions worldwide.
Students shape their own academic profiles by selecting elective modules
that match their interests and career goals, culminating in a Master's
thesis within their chosen focus area.

A mandatory internship provides practical experience and insight into
professional fields related to environmental sciences. Throughout the
program, students develop research and methodological skills, scientific
communication abilities, and the flexibility to engage with new and
complex problems. The program structure allows for a study abroad
semester, typically in the third semester, further enriching the
international and intercultural dimension of the degree.

\section*{Admission Requirements}\label{admission-requirements}
\addcontentsline{toc}{section}{Admission Requirements}

\markright{Admission Requirements}

The admissions requirements are described on the
\href{https://imes.uni-koeln.de/prospective-students/apply-for-the-imes-masters-program}{IMES
website} and can be found in the Admission Regulations.

\chapter*{Qualification Profile}\label{qualification-profile}
\addcontentsline{toc}{chapter}{Qualification Profile}

\markboth{Qualification Profile}{Qualification Profile}

\section*{The German Qualifications Framework for Higher Education
(HQR)}\label{the-german-qualifications-framework-for-higher-education-hqr}
\addcontentsline{toc}{section}{The German Qualifications Framework for
Higher Education (HQR)}

\markright{The German Qualifications Framework for Higher Education
(HQR)}

The \emph{German Qualifications Framework for Higher Education (HQR)}
describes the key competencies that students are expected to develop
during their studies at German universities. It serves as a national
reference for ensuring high academic standards, transparent learning
outcomes, and comparability across degree programs---both nationally and
internationally.

\subsection*{Why does the HQR matter?}\label{why-does-the-hqr-matter}
\addcontentsline{toc}{subsection}{Why does the HQR matter?}

The HQR plays an important role in:

\begin{itemize}
\tightlist
\item
  \textbf{Designing degree programs}: Universities use it to define what
  students should know and be able to do at different academic levels
  (Bachelor, Master, Doctorate).
\item
  \textbf{Developing modules and assessments}: It helps structure
  learning outcomes, teaching methods, and examinations.
\item
  \textbf{Ensuring international comparability}: The HQR aligns with the
  European Qualifications Framework (EQF) and the Qualifications
  Framework for the European Higher Education Area (QF-EHEA).
\item
  \textbf{Supporting mobility and recognition}: It facilitates credit
  transfer, student exchange, and mutual recognition of degrees across
  countries.
\end{itemize}

\subsection*{What are the core competence
dimensions?}\label{what-are-the-core-competence-dimensions}
\addcontentsline{toc}{subsection}{What are the core competence
dimensions?}

The HQR defines four broad competence areas that every Master's graduate
should develop:

\begin{itemize}
\item
  \textbf{Knowledge and Understanding}\\
  Deep and critical understanding of theories, methods, and debates in
  the field of study.
\item
  \textbf{Use, Application, and Generation of Knowledge}\\
  Ability to apply knowledge to new problems and to independently
  conduct research.
\item
  \textbf{Communication and Cooperation}\\
  Skills to communicate effectively, work in diverse teams, and engage
  in academic or professional discourse.
\item
  \textbf{Academic Self-Understanding and Professionalism}\\
  Capacity for ethical reflection, independent judgement, and
  responsible professional behavior.
\end{itemize}

\subsection*{What does the HQR mean for the IMES
program?}\label{what-does-the-hqr-mean-for-the-imes-program}
\addcontentsline{toc}{subsection}{What does the HQR mean for the IMES
program?}

The integration of the German Qualifications Framework for Higher
Education (HQR) into the IMES program contributes to the academic
structure, international comparability, and competence orientation of
the degree. Specifically, it ensures that:

\begin{itemize}
\tightlist
\item
  Program outcomes are competence-driven: Each module fosters academic
  and professional skills across the four HQR dimensions.
\item
  Learning objectives are transparent and structured: Students gain
  clarity about what they are expected to know and be able to do at each
  stage of the program.
\item
  The degree is internationally comparable: Alignment with the HQR and
  European frameworks supports recognition of qualifications beyond
  Germany.
\item
  Graduates are well prepared: The program equips students for both
  research careers and applied roles in environmental science, policy,
  and practice.
\item
  Quality assurance is ensured: The framework supports consistent
  standards in curriculum development, teaching, and assessment.
\end{itemize}

\section*{Qualification Profile of
IMES}\label{qualification-profile-of-imes}
\addcontentsline{toc}{section}{Qualification Profile of IMES}

\markright{Qualification Profile of IMES}

This qualification profile outlines the core competencies expected from
graduates of the IMES program. It supports curriculum development,
teaching strategies, assessment, and quality assurance. The profile is
based on the four competence dimensions of the German Qualifications
Framework for Higher Education (HQR) and adapted to the
interdisciplinary and international nature of IMES.

\subsection*{1. Knowledge and
Understanding}\label{knowledge-and-understanding}
\addcontentsline{toc}{subsection}{1. Knowledge and Understanding}

Graduates of the IMES program possess:

\begin{itemize}
\item
  Broad interdisciplinary knowledge of environmental systems,
  interactions between natural and human systems, and global
  environmental challenges.
\item
  Specialized understanding of theories, methods, and key debates in at
  least two relevant disciplinary fields (e.g., environmental economics,
  environmental law, geosciences, physical or human geography, ecology,
  or political science).
\item
  A critical awareness of the limits of knowledge and current research
  trends in environmental sciences, including their societal relevance.
\end{itemize}

\subsection*{2. Use, Application, and Generation of
Knowledge}\label{use-application-and-generation-of-knowledge}
\addcontentsline{toc}{subsection}{2. Use, Application, and Generation of
Knowledge}

Graduates are able to:

\begin{itemize}
\tightlist
\item
  Apply advanced methods of data collection, analysis, and modeling to
  environmental problems.
\item
  Integrate and synthesize scientific, legal, economic, and social
  perspectives in problem-solving.
\item
  Design and conduct independent research projects, including framing of
  research questions, selection of appropriate methods, and
  interpretation of results.
\item
  Develop innovative and context-sensitive solutions to complex,
  real-world environmental challenges.
\end{itemize}

\subsection*{3. Communication and
Cooperation}\label{communication-and-cooperation}
\addcontentsline{toc}{subsection}{3. Communication and Cooperation}

Graduates are able to:

\begin{itemize}
\tightlist
\item
  Effectively communicate scientific and policy-relevant insights to
  diverse audiences, including stakeholders from science, policy, and
  society.
\item
  Work in intercultural, interdisciplinary teams and contribute to
  collaborative decision-making processes.
\item
  Navigate conflicts of interest and differing worldviews with empathy,
  professionalism, and ethical sensitivity.
\end{itemize}

\subsection*{4. Academic Self-Understanding and
Professionalism}\label{academic-self-understanding-and-professionalism}
\addcontentsline{toc}{subsection}{4. Academic Self-Understanding and
Professionalism}

Graduates are able to:

\begin{itemize}
\tightlist
\item
  Reflect on their academic and professional roles in society and the
  ethical implications of their work.
\item
  Are prepared for careers in science, policy-making, consultancy, NGOs,
  or further academic research (PhD).
\item
  Show commitment to life-long learning and critical self-reflection in
  complex, changing environments.
\end{itemize}

\chapter*{Program Structure}\label{program-structure}
\addcontentsline{toc}{chapter}{Program Structure}

\markboth{Program Structure}{Program Structure}

\section*{Thematic Categories of the IMES
Curriculum}\label{thematic-categories-of-the-imes-curriculum}
\addcontentsline{toc}{section}{Thematic Categories of the IMES
Curriculum}

\markright{Thematic Categories of the IMES Curriculum}

The IMES program is organized into modules---self-contained units of
study that integrate related topics into coherent blocks of learning.
Each module carries a defined number of ECTS credits and includes
clearly articulated learning outcomes and assessments. Most modules are
designed to be completed within a single semester, though some may
extend across two.

To ensure a broad and interdisciplinary foundation, IMES modules are
grouped into five thematic categories. Students are required to complete
\textbf{at least 6 ECTS credits in each category} as part of the degree
requirements:

\begin{longtable}[]{@{}
  >{\raggedright\arraybackslash}p{(\columnwidth - 4\tabcolsep) * \real{0.0882}}
  >{\raggedright\arraybackslash}p{(\columnwidth - 4\tabcolsep) * \real{0.2647}}
  >{\raggedright\arraybackslash}p{(\columnwidth - 4\tabcolsep) * \real{0.6471}}@{}}
\toprule\noalign{}
\begin{minipage}[b]{\linewidth}\raggedright
\textbf{Code}
\end{minipage} & \begin{minipage}[b]{\linewidth}\raggedright
\textbf{Category Name}
\end{minipage} & \begin{minipage}[b]{\linewidth}\raggedright
\textbf{Topics include, but are not limited to}
\end{minipage} \\
\midrule\noalign{}
\endhead
\bottomrule\noalign{}
\endlastfoot
A & Earth and Environmental Systems & Physical environment, natural
sciences, climate, ecosystems \\
B & Societies, Education, and Ethics & Human values, education for
sustainability, ethics, cities, resilience \\
C & Governance, Law, and Institutions & Environmental law, political
systems, regulation, institutions \\
D & Economy and Resource Management & Environmental economics, renewable
energy, resource management \\
E & Research and Analytical Methods & GIS, statistics, modeling,
interdisciplinary methods, research methods \\
\end{longtable}

\section*{Overview of Module Types in
IMES}\label{overview-of-module-types-in-imes}
\addcontentsline{toc}{section}{Overview of Module Types in IMES}

\markright{Overview of Module Types in IMES}

The IMES program comprises modules that are typically worth either 6 or
12 ECTS. The Master Thesis and Colloquium form a comprehensive research
module worth 30 ECTS. The curriculum includes six mandatory modules and
a wide range of electives, allowing students to shape their individual
academic trajectories.

Modules in IMES are classified into Core Modules, Advanced Modules and
Specialisation Modules (\textbf{?@sec-concepts}). A detailed
descriptions of each module are provided in the second part of this
module handbook.

\begin{longtable}[]{@{}
  >{\raggedright\arraybackslash}p{(\columnwidth - 10\tabcolsep) * \real{0.2941}}
  >{\raggedright\arraybackslash}p{(\columnwidth - 10\tabcolsep) * \real{0.0980}}
  >{\raggedright\arraybackslash}p{(\columnwidth - 10\tabcolsep) * \real{0.1569}}
  >{\raggedright\arraybackslash}p{(\columnwidth - 10\tabcolsep) * \real{0.1569}}
  >{\raggedright\arraybackslash}p{(\columnwidth - 10\tabcolsep) * \real{0.0980}}
  >{\raggedright\arraybackslash}p{(\columnwidth - 10\tabcolsep) * \real{0.1961}}@{}}
\caption{List of modules in IMES. Module types are Core Modules (CM),
Advanced Modules (AM) and Specialisation modules (SM)}\tabularnewline
\toprule\noalign{}
\begin{minipage}[b]{\linewidth}\raggedright
\textbf{Module}
\end{minipage} & \begin{minipage}[b]{\linewidth}\raggedright
\textbf{ECTS}
\end{minipage} & \begin{minipage}[b]{\linewidth}\raggedright
\textbf{Mandatory (M) or Elective (E)}
\end{minipage} & \begin{minipage}[b]{\linewidth}\raggedright
\textbf{Module Type}
\end{minipage} & \begin{minipage}[b]{\linewidth}\raggedright
\textbf{Semester}
\end{minipage} & \begin{minipage}[b]{\linewidth}\raggedright
\textbf{Category}
\end{minipage} \\
\midrule\noalign{}
\endfirsthead
\toprule\noalign{}
\begin{minipage}[b]{\linewidth}\raggedright
\textbf{Module}
\end{minipage} & \begin{minipage}[b]{\linewidth}\raggedright
\textbf{ECTS}
\end{minipage} & \begin{minipage}[b]{\linewidth}\raggedright
\textbf{Mandatory (M) or Elective (E)}
\end{minipage} & \begin{minipage}[b]{\linewidth}\raggedright
\textbf{Module Type}
\end{minipage} & \begin{minipage}[b]{\linewidth}\raggedright
\textbf{Semester}
\end{minipage} & \begin{minipage}[b]{\linewidth}\raggedright
\textbf{Category}
\end{minipage} \\
\midrule\noalign{}
\endhead
\bottomrule\noalign{}
\endlastfoot
Introduction to Natural and Social Environmental Sciences & 6 & M & CM &
1 & Research and Analytical Methods \\
Geosphere and Hydrosphere & 6 & E & AM & 1 & Earth and Environmental
Systems \\
Sustainable Development & 6 & E & AM & 1 & Societies, Education, and
Ethics \\
Environmental Law & 6 & M & AM & 1 and 2 & Governance, Law, and
Institutions \\
Energy and Climate Change & 6 & E & AM & 1 or 2 & Economy and Resource
Management \\
Environmental Medicine & 6 & E & AM & 1 & Societies, Education, and
Ethics \\
Anthropology & 6 & E & AM & 1 and 2 & Societies, Education, and
Ethics \\
Methods of Spatial and Statistical Data Analysis & 6 & E & AM & 1 and 2
& Research and Analytical Methods \\
Environmental Humanities and Communication & 6 & E & AM & 1 & Societies,
Education, and Ethics \\
Advanced Environmental Sciences & 6 & M & AM & 2 & Research and
Analytical Methods \\
Ecosystems and Landscape Dynamics & 6 & M & AM & 2 & Earth and
Environmental Systems \\
Natural Resources, Water and Renewable Energy Management (TH Köln) & 6 &
E & AM & 2 or 3 & Economy and Resource Management \\
Environmental Ethics and Management & 6 & E & AM & 2 & Societies,
Education, and Ethics \\
Meteorology & 6 & E & AM & 2 & Earth and Environmental Systems \\
Environmental Modelling and Data Science & 6 & E & AM & 2 or 3 &
Research and Analytical Methods \\
Environmental Pollution & 6 & E & AM & 2 & Earth and Environmental
Systems \\
Internship & 12 & M & SM & 3 & Research and Analytical Methods \\
Advanced Atmospheric Processes & 6 & E & AM & 3 & Earth and
Environmental Systems \\
Human Environment Relations & 6 & E & AM & 3 & Societies, Education, and
Ethics \\
Advanced Environmental Geography & 6 & E & AM & 3 & Earth and
Environmental Systems \\
Environmental Spatial Methods & 6 & E & AM & 3 & Research and Analytical
Methods \\
Individual Specialization Module & 6 & E & SM & 3 & Research and
Analytical Methods \\
Individual Advanced Module & 6 & E & AM & 3 & Any \\
Master Thesis and Colloquium & 30 & M & & 4 & Any \\
\end{longtable}

\part{MODULE DESCRIPTIONS}

\chapter*{Geosphere and Hydrosphere}\label{geosphere-and-hydrosphere}
\addcontentsline{toc}{chapter}{Geosphere and Hydrosphere}

\markboth{Geosphere and Hydrosphere}{Geosphere and Hydrosphere}

\begin{longtable}[]{@{}ll@{}}
\toprule\noalign{}
\endhead
\bottomrule\noalign{}
\endlastfoot
\textbf{Type of Module} & Advanced Module \\
\textbf{Module Code} & IMES-AM-GeoHyd \\
\textbf{Workload} & 180 h \\
\textbf{ECTS} & 6 \\
\textbf{Term} & Semester 1 \\
\textbf{Offered Every} & WiSe \\
\textbf{Start} & WiSe \\
\textbf{Duration} & 1 semester \\
\end{longtable}

\begin{longtable}[]{@{}
  >{\raggedright\arraybackslash}p{(\columnwidth - 4\tabcolsep) * \real{0.3111}}
  >{\raggedright\arraybackslash}p{(\columnwidth - 4\tabcolsep) * \real{0.3111}}
  >{\raggedright\arraybackslash}p{(\columnwidth - 4\tabcolsep) * \real{0.3778}}@{}}
\toprule\noalign{}
\begin{minipage}[b]{\linewidth}\raggedright
Course Types
\end{minipage} & \begin{minipage}[b]{\linewidth}\raggedright
Contact Time
\end{minipage} & \begin{minipage}[b]{\linewidth}\raggedright
Self-study Time
\end{minipage} \\
\midrule\noalign{}
\endhead
\bottomrule\noalign{}
\endlastfoot
a) Lecture: Introduction to Environmental Geophysics & 2 SWS / 30 h & 60
h \\
b) Lecture: Physical Hydrology & 2 SWS / 30 h & 60 h \\
\end{longtable}

\section*{Module Description}\label{module-description}
\addcontentsline{toc}{section}{Module Description}

\markright{Module Description}

This module consists of two lectures: \emph{Introduction to
Environmental Geophysics} and \emph{Physical Hydrology}. It provides a
comprehensive introduction to the principles and techniques of
geophysics and hydrology, equipping students with the foundational
knowledge to address environmental and water resource challenges.
Students will explore geophysical methods such as electromagnetic and
seismic techniques, emphasizing techniques essential for investigating
the subsurface in environmental science. Additionally, the module covers
essential hydrological concepts, including water fluxes, storage
processes, and key methods for water resource management, enabling
students to evaluate sustainable water use strategies effectively.
Together, these two lectures introduce students to the study of the
critical zone, the thin skin of the Earth's surface.

\section*{Module Objectives and
Outcomes}\label{module-objectives-and-outcomes}
\addcontentsline{toc}{section}{Module Objectives and Outcomes}

\markright{Module Objectives and Outcomes}

\subsection*{a) Lecture: Introduction to Environmental
Geophysics}\label{a-lecture-introduction-to-environmental-geophysics}
\addcontentsline{toc}{subsection}{a) Lecture: Introduction to
Environmental Geophysics}

The main objective of this lecture is to introduce students to
fundamental concepts, methodologies, and basic interpretative techniques
of geophysical methods applied in environmental studies. Building on
this foundation, students will achieve the following learning outcomes:

\begin{itemize}
\tightlist
\item
  Identify and describe basic geophysical methods commonly used in
  environmental science.
\item
  Understand fundamental principles and physical concepts underlying
  geophysical exploration techniques, particularly electromagnetic and
  seismic methods.
\item
  Explain how geophysical data can be used to analyze environmental
  conditions and assess basic environmental risks.
\item
  Recognize the strengths and limitations of various geophysical
  techniques in environmental applications.
\end{itemize}

\subsection*{b) Lecture: Physical
Hydrology}\label{b-lecture-physical-hydrology}
\addcontentsline{toc}{subsection}{b) Lecture: Physical Hydrology}

The primary objective of this lecture is to introduce students to the
foundational concepts of hydrology and basic methods used in water
resource management, preparing them to assess the potential and
limitations of different water uses. Upon completing this module,
students will be able to:

\begin{itemize}
\tightlist
\item
  Explain the fundamental concepts of hydrology, including water fluxes
  and storage processes.
\item
  Identify key hydrological methods used in water resource management.
\item
  Assess the potential and limitations of various water uses,
  considering the governing processes of water fluxes and storage.
\item
  Discuss basic hydrological processes and their implications for water
  resource management.
\end{itemize}

\section*{Module Content}\label{module-content}
\addcontentsline{toc}{section}{Module Content}

\markright{Module Content}

\subsection*{a) Lecture: Introduction to Environmental
Geophysics}\label{a-lecture-introduction-to-environmental-geophysics-1}
\addcontentsline{toc}{subsection}{a) Lecture: Introduction to
Environmental Geophysics}

\begin{itemize}
\tightlist
\item
  Overview of primary geophysical exploration methods and their
  significance in environmental studies.
\item
  Fundamental physical principles underlying electromagnetic and seismic
  techniques.
\item
  Methodologies and interpretative procedures used in geophysical data
  analysis.
\item
  Applications of geophysical methods for environmental risk assessment
  and subsurface investigation.
\end{itemize}

\subsection*{b) Lecture: Physical
Hydrology}\label{b-lecture-physical-hydrology-1}
\addcontentsline{toc}{subsection}{b) Lecture: Physical Hydrology}

\begin{itemize}
\tightlist
\item
  Components and processes of the water cycle across various spatial
  scales.
\item
  Water fluxes and storage terms on land surfaces and their measurement.
\item
  Environmental factors influencing water availability and distribution.
\item
  Fundamental methods and concepts to describe, measure and model water
  fluxes.
\item
  Basic principles and challenges of water resource management in the
  context of environmental sustainability.
\end{itemize}

\section*{Teaching Methods}\label{teaching-methods}
\addcontentsline{toc}{section}{Teaching Methods}

\markright{Teaching Methods}

Lecture

\section*{Prerequisites}\label{prerequisites}
\addcontentsline{toc}{section}{Prerequisites}

\markright{Prerequisites}

None

\section*{Type of Examination}\label{type-of-examination}
\addcontentsline{toc}{section}{Type of Examination}

\markright{Type of Examination}

Written examination of 90--120 minutes. The subject of examination is
the content of the lectures (a and b) of this module description.

\section*{Credits Awarded}\label{credits-awarded}
\addcontentsline{toc}{section}{Credits Awarded}

\markright{Credits Awarded}

Credit points are awarded upon successful completion of the module's
examination, with a minimum passing grade of 4.0 (German grading
system).

\section*{Compatibility with Other
Curricula}\label{compatibility-with-other-curricula}
\addcontentsline{toc}{section}{Compatibility with Other Curricula}

\markright{Compatibility with Other Curricula}

JIMES (JIMES-AM-GeoHyd)

\section*{Proportion of Final Grade}\label{proportion-of-final-grade}
\addcontentsline{toc}{section}{Proportion of Final Grade}

\markright{Proportion of Final Grade}

5\%

\section*{Module Coordinator}\label{module-coordinator}
\addcontentsline{toc}{section}{Module Coordinator}

\markright{Module Coordinator}

Prof.~Dr.~Karl Schneider (Institute of Geography)

\section*{Further Information}\label{further-information}
\addcontentsline{toc}{section}{Further Information}

\markright{Further Information}

There are no restrictions on the number of places available for IMES
students.

\chapter*{Sustainable Development}\label{sustainable-development}
\addcontentsline{toc}{chapter}{Sustainable Development}

\markboth{Sustainable Development}{Sustainable Development}

\begin{longtable}[]{@{}ll@{}}
\toprule\noalign{}
\endhead
\bottomrule\noalign{}
\endlastfoot
\textbf{Type of Module} & Advanced Module \\
\textbf{Module Code} & IMES-AM-SD \\
\textbf{Workload} & 180 h \\
\textbf{ECTS} & 6 \\
\textbf{Term} & Semester \\
\textbf{Offered Every} & WiSe \\
\textbf{Start} & WiSe \\
\textbf{Duration} & 1 semester \\
\end{longtable}

\begin{longtable}[]{@{}lll@{}}
\toprule\noalign{}
Course Types & Contact Time & Self-study Time \\
\midrule\noalign{}
\endhead
\bottomrule\noalign{}
\endlastfoot
a) Seminar: Sustainable Development & 2 SWS / 30 h & 60 h \\
b) Practical: Fieldwork & 2 SWS / 30 h & 60 h \\
\end{longtable}

\section*{Module Description}\label{module-description-1}
\addcontentsline{toc}{section}{Module Description}

\markright{Module Description}

This module consists of two courses: a literature-based seminar in
Sustainable Development and Fieldwork. It provides an in-depth
exploration of the principles, challenges, and frameworks of
sustainability. Key topics include the Sustainable Development Goals
(SDGs), climate change, planetary boundaries, loss of biodiversity and
Education for Sustainable Action (ESA) as a tool for transformative
action. The course addresses critical global issues such as inequality,
urbanization, and environmental degradation, emphasizing their
interconnections and linking them to actionable strategies for
sustainable solutions.

\section*{Module Objectives}\label{module-objectives}
\addcontentsline{toc}{section}{Module Objectives}

\markright{Module Objectives}

\begin{itemize}
\tightlist
\item
  Explore the foundational concepts of sustainable development,
  including its pillars, challenges, and global frameworks like Agenda
  21 and the SDGs.
\item
  Analyse further important concepts like planetary boundaries, climate
  change and loss of biodiversity, based on scientific literature.
\item
  Examine the role of Education for Sustainable Action (ESA) in
  promoting transformative learning, systems thinking, and
  sustainability action.
\end{itemize}

\section*{Module Outcomes}\label{module-outcomes}
\addcontentsline{toc}{section}{Module Outcomes}

\markright{Module Outcomes}

\begin{itemize}
\tightlist
\item
  Understand the foundational principles of sustainable development,
  including its pillars, challenges, and key global frameworks such as
  Agenda 21 and the SDGs.
\item
  Analyze complex concepts like planetary boundaries, climate change,
  and biodiversity loss by critically engaging with scientific
  literature.
\item
  Evaluate the challenges and opportunities in achieving sustainable
  development in the context of global and local issues.
\item
  Examine the role of Education for Sustainable Development (ESD) in
  fostering transformative learning, systems thinking, and actionable
  strategies.
\item
  Develop practical skills in applying systems thinking to identify and
  address sustainability challenges during Field Work.
\end{itemize}

\section*{Module Content}\label{module-content-1}
\addcontentsline{toc}{section}{Module Content}

\markright{Module Content}

\subsection*{a) Seminar: Sustainable
Development}\label{a-seminar-sustainable-development}
\addcontentsline{toc}{subsection}{a) Seminar: Sustainable Development}

\begin{itemize}
\tightlist
\item
  Basic concepts of sustainable development:

  \begin{itemize}
  \tightlist
  \item
    Understanding `sustainability' and `development'
  \item
    Causes and effects of unsustainable development like consumerism,
    globalization, urbanization, ecosystem degradation, inequity
  \item
    Pillars of sustainable development: economic, social, environmental,
    political (governance)
  \item
    Timeline of sustainable development; Agenda 21; Millennium
    development goals and sustainable development goals.
  \item
    ESD as a tool for transformative learning and systems thinking
  \end{itemize}
\item
  Key sustainability concepts and challenges:

  \begin{itemize}
  \tightlist
  \item
    Climate change: causes, impacts, and mitigation strategies
  \item
    Planetary boundaries and their relevance to sustainability
  \item
    Ecosystem services and their role in sustainable development
  \item
    Rapid population growth and food security
  \item
    Poverty, income inequality, and gender disparity
  \item
    Urbanization and its challenges, including energy transformation
  \item
    Environmental degradation and loss of biodiversity
  \end{itemize}
\end{itemize}

\subsection*{b) Practical: Fieldwork}\label{b-practical-fieldwork}
\addcontentsline{toc}{subsection}{b) Practical: Fieldwork}

\begin{itemize}
\tightlist
\item
  Visit and analysis of sustainability projects.
\item
  Developing a matrix for analysing sustainability projects along with
  indicators.
\end{itemize}

\section*{Teaching Methods}\label{teaching-methods-1}
\addcontentsline{toc}{section}{Teaching Methods}

\markright{Teaching Methods}

Literature-based seminar and practical

\section*{Prerequisites}\label{prerequisites-1}
\addcontentsline{toc}{section}{Prerequisites}

\markright{Prerequisites}

None

\section*{Type of Examination}\label{type-of-examination-1}
\addcontentsline{toc}{section}{Type of Examination}

\markright{Type of Examination}

Portfolio based on a) and b), 5 to 10 pages.

\section*{Credits Awarded}\label{credits-awarded-1}
\addcontentsline{toc}{section}{Credits Awarded}

\markright{Credits Awarded}

Credit points are awarded upon regular and active participation, as well
as the successful completion of the module's examination with a minimum
passing grade of 4.0. (sufficient).

\section*{Compatibility with Other
Curricula}\label{compatibility-with-other-curricula-1}
\addcontentsline{toc}{section}{Compatibility with Other Curricula}

\markright{Compatibility with Other Curricula}

JIMES (JIMES-AM-SDUoC), M.Sc. Module part (a) in Geography (AM1 and
AM2), MA Geography (AM1 and AM2) and MEd Geography (AM1)

\section*{Proportion of Final Grade}\label{proportion-of-final-grade-1}
\addcontentsline{toc}{section}{Proportion of Final Grade}

\markright{Proportion of Final Grade}

5\%

\section*{Module Coordinator}\label{module-coordinator-1}
\addcontentsline{toc}{section}{Module Coordinator}

\markright{Module Coordinator}

Dr.~Veronika Selbach and Dr.~Verena Dlugoß (Institute of Geography)

\section*{Further Information}\label{further-information-1}
\addcontentsline{toc}{section}{Further Information}

\markright{Further Information}

Restrictions on the number of places available for IMES students might
apply.

\chapter*{Environmental Law}\label{environmental-law}
\addcontentsline{toc}{chapter}{Environmental Law}

\markboth{Environmental Law}{Environmental Law}

\begin{longtable}[]{@{}ll@{}}
\toprule\noalign{}
\endhead
\bottomrule\noalign{}
\endlastfoot
\textbf{Type of Module} & Advanced Module \\
\textbf{Module Code} & IMES-AM-Law \\
\textbf{Workload} & 180 h \\
\textbf{ECTS} & 6 \\
\textbf{Term} & Semester 1 and 2 \\
\textbf{Offered Every} & WiSe / SuSe \\
\textbf{Start} & WiSe \\
\textbf{Duration} & 2 semesters \\
\end{longtable}

\begin{longtable}[]{@{}
  >{\raggedright\arraybackslash}p{(\columnwidth - 4\tabcolsep) * \real{0.4000}}
  >{\raggedright\arraybackslash}p{(\columnwidth - 4\tabcolsep) * \real{0.3000}}
  >{\raggedright\arraybackslash}p{(\columnwidth - 4\tabcolsep) * \real{0.3000}}@{}}
\toprule\noalign{}
\begin{minipage}[b]{\linewidth}\raggedright
Course Types
\end{minipage} & \begin{minipage}[b]{\linewidth}\raggedright
Contact Time
\end{minipage} & \begin{minipage}[b]{\linewidth}\raggedright
Self-study Time
\end{minipage} \\
\midrule\noalign{}
\endhead
\bottomrule\noalign{}
\endlastfoot
a) Lecture: Environmental Law: Comparative and Basic Studies & 2 SWS 30
h & 60 h \\
b) Lecture: International Environmental Law & 2 SWS 30 h & 60 h \\
\end{longtable}

\section*{Module Description}\label{module-description-2}
\addcontentsline{toc}{section}{Module Description}

\markright{Module Description}

The module consists of two lectures: \emph{Environmental Law:
Comparative and Basic Studies} and \emph{International Environmental
Law}. Environmental Law introduces students to the discipline of law,
with a focus on domestic legal systems and the role of environmental law
within these systems. It explores how environmental problems have been
addressed through legal mechanisms, both historically and in
contemporary contexts, including before and after the introduction of
dedicated environmental legislation.

The lecture \emph{International Environmental Law} advances students'
understanding of domestic law introduced in lecture a), focusing on
international law and its application to global environmental
challenges. Students will explore the unique sources, institutions, and
subjects of international environmental law, and distinguish them from
their domestic counterparts. The course emphasizes the role of treaties,
litigation, and arbitration in shaping international environmental
governance.

\section*{Module Objectives}\label{module-objectives-1}
\addcontentsline{toc}{section}{Module Objectives}

\markright{Module Objectives}

\begin{itemize}
\tightlist
\item
  Introduce students to the principles, sources, institutions, and
  subjects of domestic and international environmental law.
\item
  Develop an understanding of how environmental challenges are addressed
  through domestic and international legal mechanisms.
\item
  Highlight the role of treaties, litigation, and arbitration in shaping
  international environmental governance.
\item
  Explore the interdisciplinary nature of environmental law and its
  connections with social and natural sciences.
\end{itemize}

\section*{Module Outcomes}\label{module-outcomes-1}
\addcontentsline{toc}{section}{Module Outcomes}

\markright{Module Outcomes}

\begin{itemize}
\tightlist
\item
  Identify and evaluate key sources, principles, and institutions of
  domestic and international environmental law.
\item
  Compare domestic and international approaches to addressing
  environmental problems, including their advantages and limitations.
\item
  Analyze the role of legal mechanisms in environmental governance and
  their interplay with social and natural sciences.
\item
  Apply knowledge of international environmental law to evaluate global
  environmental challenges and propose solutions.
\end{itemize}

\section*{Module Content}\label{module-content-2}
\addcontentsline{toc}{section}{Module Content}

\markright{Module Content}

\subsection*{a) Lecture: Environmental Law: Comparative and Basic
Studies}\label{a-lecture-environmental-law-comparative-and-basic-studies}
\addcontentsline{toc}{subsection}{a) Lecture: Environmental Law:
Comparative and Basic Studies}

\begin{itemize}
\tightlist
\item
  The placement of environmental law within domestic legal systems.
\item
  Domestic sources of law, key institutions, and subjects relevant to
  environmental governance.
\item
  The concept of sources of law, focusing on the fundamentals of
  identifying and locating applicable laws to address specific
  environmental challenges.
\item
  How legal principles and processes contribute to solving environmental
  problems at the domestic level.
\item
  An overview of the comparative method to understand differences in how
  environmental law is applied across jurisdictions.
\item
  Interdisciplinary links between environmental law and other fields,
  including the social and natural sciences.
\end{itemize}

\subsection*{b) Lecture: International Environmental
Law}\label{b-lecture-international-environmental-law}
\addcontentsline{toc}{subsection}{b) Lecture: International
Environmental Law}

\begin{itemize}
\tightlist
\item
  The distinction between international public environmental law and
  political processes.
\item
  Fundamental principles of international environmental legislation,
  including its interpretation and execution.
\item
  International sources of law, key institutions, and subjects, with a
  focus on their unique roles and functions.
\item
  Comparative analysis of international and domestic environmental law,
  highlighting their differences and intersections.
\item
  Study of treaties categorized by natural science domains: atmosphere,
  hydrosphere, and geosphere.
\item
  Examination of litigation and arbitration cases to identify
  international customs and principles.
\item
  Analysis of the interdisciplinary connections between international
  environmental law and the social and natural sciences.
\end{itemize}

\section*{Teaching Methods}\label{teaching-methods-2}
\addcontentsline{toc}{section}{Teaching Methods}

\markright{Teaching Methods}

Lecture

\section*{Prerequisites}\label{prerequisites-2}
\addcontentsline{toc}{section}{Prerequisites}

\markright{Prerequisites}

\begin{enumerate}
\def\labelenumi{\alph{enumi})}
\tightlist
\item
  None\\
\item
  While there is no formal requirement, International Environmental Law
  will be taught assuming the student has a basic understanding of
  sources of law, institutions of law and subjects of law, however that
  understanding was achieved.
\end{enumerate}

\section*{Type of Examination}\label{type-of-examination-2}
\addcontentsline{toc}{section}{Type of Examination}

\markright{Type of Examination}

Students are required to independently research and write a paper in
English, comprising 20--24 pages, to be completed outside the lecture
hours.

\textbf{Part I} of the research paper will address an environmental
problem in a country of their choice, presents that country's official
state policy on the problem, presents the sources and institutions of
law on that problem in that country, and then assesses whether those
laws and institutions solve the problem. The student will then do the
same for a second country of choice. The paper will then build a matrix
by which one can compare the performance of the first country's
environmental solutions with the second.

\textbf{Part II} of the research paper will address an analysis of a
case study of international environmental law.

\section*{Credits Awarded}\label{credits-awarded-2}
\addcontentsline{toc}{section}{Credits Awarded}

\markright{Credits Awarded}

Credit points are awarded upon successful completion of the module's
examination, with a minimum passing grade of 4.0 (sufficient).

\section*{Compatibility with Other
Curricula}\label{compatibility-with-other-curricula-2}
\addcontentsline{toc}{section}{Compatibility with Other Curricula}

\markright{Compatibility with Other Curricula}

JIMES (JIMES-AM-Law)

\section*{Proportion of Final Grade}\label{proportion-of-final-grade-2}
\addcontentsline{toc}{section}{Proportion of Final Grade}

\markright{Proportion of Final Grade}

5\%

\section*{Module Coordinator}\label{module-coordinator-2}
\addcontentsline{toc}{section}{Module Coordinator}

\markright{Module Coordinator}

Prof.~Dr.~Kirk Junker (US American Law)

\section*{Further Information}\label{further-information-2}
\addcontentsline{toc}{section}{Further Information}

\markright{Further Information}

There are no restrictions on the number of places available for IMES
students.

\chapter*{Ecosystems and Landscape
Dynamics}\label{ecosystems-and-landscape-dynamics}
\addcontentsline{toc}{chapter}{Ecosystems and Landscape Dynamics}

\markboth{Ecosystems and Landscape Dynamics}{Ecosystems and Landscape
Dynamics}

\begin{longtable}[]{@{}ll@{}}
\toprule\noalign{}
\endhead
\bottomrule\noalign{}
\endlastfoot
\textbf{Type of Module} & Advanced Module \\
\textbf{Module Code} & IMES-AM-Ecosys \\
\textbf{Workload} & 180 h \\
\textbf{ECTS} & 6 \\
\textbf{Term} & Semester 3 \\
\textbf{Offered Every} & SuSe \\
\textbf{Start} & SuSe \\
\textbf{Duration} & 1 semester \\
\end{longtable}

\begin{longtable}[]{@{}
  >{\raggedright\arraybackslash}p{(\columnwidth - 4\tabcolsep) * \real{0.3111}}
  >{\raggedright\arraybackslash}p{(\columnwidth - 4\tabcolsep) * \real{0.3111}}
  >{\raggedright\arraybackslash}p{(\columnwidth - 4\tabcolsep) * \real{0.3778}}@{}}
\toprule\noalign{}
\begin{minipage}[b]{\linewidth}\raggedright
Course Types
\end{minipage} & \begin{minipage}[b]{\linewidth}\raggedright
Contact Time
\end{minipage} & \begin{minipage}[b]{\linewidth}\raggedright
Self-study Time
\end{minipage} \\
\midrule\noalign{}
\endhead
\bottomrule\noalign{}
\endlastfoot
a) Lecture: Ecosystem Services and Functions Under Climate Change & 2
SWS / 30 h & 60 h \\
b) Lecture: Landscape Formation & 2 SWS / 30 h & 60 h \\
\end{longtable}

\section*{Module Description}\label{module-description-3}
\addcontentsline{toc}{section}{Module Description}

\markright{Module Description}

This module examines the interactions between ecosystems, landscapes,
and human impacts, providing a foundation for understanding
environmental processes and their role in sustainable development. The
lecture \emph{Ecosystem Services and Functions Under Climate Change}
introduces key concepts from soil science, soil physics, and plant
nutrition, focusing on their roles within ecosystem services and
functions. It expands students' understanding of the complex processes
in ecosystems and the influence of land \ldots{}

The lecture \emph{Landscape Formation} focuses on the factors, dynamics,
and outputs of landscape evolution, with special regard to the
Quaternary period and the interaction between natural phenomena and
human activities.

\section*{Module Objectives}\label{module-objectives-2}
\addcontentsline{toc}{section}{Module Objectives}

\markright{Module Objectives}

\subsection*{a) Lecture: Ecosystem Services and Functions Under Climate
Change}\label{a-lecture-ecosystem-services-and-functions-under-climate-change}
\addcontentsline{toc}{subsection}{a) Lecture: Ecosystem Services and
Functions Under Climate Change}

\begin{itemize}
\tightlist
\item
  Introduce the processes within ecosystems and their interactions with
  human land use and climate change.
\item
  Explore the key concepts of ecosystem services and their role in
  sustainable development.
\end{itemize}

\subsection*{b) Lecture: Landscape
Formation}\label{b-lecture-landscape-formation}
\addcontentsline{toc}{subsection}{b) Lecture: Landscape Formation}

\begin{itemize}
\tightlist
\item
  Examine the key factors, processes, and dynamics involved in natural
  landscape formation and evolution.
\item
  Analyze the impacts of climate change and human activities on
  geomorphological processes and landscape changes.
\end{itemize}

\section*{Module Outcomes}\label{module-outcomes-2}
\addcontentsline{toc}{section}{Module Outcomes}

\markright{Module Outcomes}

\subsection*{a) Lecture: Ecosystem Services and Functions Under Climate
Change}\label{a-lecture-ecosystem-services-and-functions-under-climate-change-1}
\addcontentsline{toc}{subsection}{a) Lecture: Ecosystem Services and
Functions Under Climate Change}

\begin{itemize}
\tightlist
\item
  Describe the processes within ecosystems and how they are affected by
  human land use change and climate change.
\item
  Explain key concepts in ecosystem services, including their role in
  sustainable development.
\item
  Analyze one selected topic related to ecosystem services in depth.
\item
  Apply the SQ4R (Survey, Question, Read, Reflect, Recite and Review)
  method to critically engage with scientific literature on ecosystem
  services and processes.
\item
  Apply AI-assisted tools for literature review and analysis.
\end{itemize}

\subsection*{b) Lecture: Landscape
Formation}\label{b-lecture-landscape-formation-1}
\addcontentsline{toc}{subsection}{b) Lecture: Landscape Formation}

\begin{itemize}
\tightlist
\item
  Identify key factors and processes involved in natural landscape
  formation, including morphography, morphometry, morphodynamics, and
  morphochronology.
\item
  Describe the dynamics of landscape evolution, particularly during the
  Quaternary period, and recognize the role of relief forms.
\item
  Explain the impact of climate change on geomorphological processes,
  highlighting how it influences events such as floods and landslides.
\item
  Discuss the effects of human activities on landscape formation,
  including the development of technical landforms.
\item
  Analyze specific geomorphological case studies to assess the
  interaction between natural processes and anthropogenic influences on
  landscape change.
\end{itemize}

\section*{Module Content}\label{module-content-3}
\addcontentsline{toc}{section}{Module Content}

\markright{Module Content}

\subsection*{a) Lecture: Ecosystem Services and Functions Under Climate
Change}\label{a-lecture-ecosystem-services-and-functions-under-climate-change-2}
\addcontentsline{toc}{subsection}{a) Lecture: Ecosystem Services and
Functions Under Climate Change}

\begin{itemize}
\tightlist
\item
  Introduction to soil science and soil hydrology
\item
  Ecosystem services and Sustainable Development Goals (SDGs)
\item
  Plant-soil interactions and nutrient cycling
\item
  Soil organic matter and carbon sequestration in soils
\item
  Soil degradation, sustainable land management and food production
\end{itemize}

\subsection*{b) Lecture: Landscape
Formation}\label{b-lecture-landscape-formation-2}
\addcontentsline{toc}{subsection}{b) Lecture: Landscape Formation}

\begin{itemize}
\tightlist
\item
  Key aspects of relief forms, focusing on morphography, morphometry,
  morphodynamics, and morphochronology as fundamental controls in
  landscape structure.
\item
  Core dynamics and processes involved in natural landscape formation
  over time.
\item
  Influence of human activities, such as the creation of technical
  landforms, on landscape modification.
\item
  Examination of how climate change intensifies geomorphological events,
  including floods and landslides.
\end{itemize}

\section*{Teaching Methods}\label{teaching-methods-3}
\addcontentsline{toc}{section}{Teaching Methods}

\markright{Teaching Methods}

Lecture

\section*{Prerequisites}\label{prerequisites-3}
\addcontentsline{toc}{section}{Prerequisites}

\markright{Prerequisites}

None

\section*{Type of Examination}\label{type-of-examination-3}
\addcontentsline{toc}{section}{Type of Examination}

\markright{Type of Examination}

Examination in form of a portfolio (5--10 pages). The subject of
examination is the content of the lecture a) of this module. During the
lecture b) of this module, students are required to complete mandatory
ungraded assignments (Studienleistungen).

\section*{Credits Awarded}\label{credits-awarded-3}
\addcontentsline{toc}{section}{Credits Awarded}

\markright{Credits Awarded}

Credit points are awarded when the portfolio has been graded with at
least 4.0 (sufficient) and the mandatory assignments (Studienleistungen)
have been successfully completed.

\section*{Compatibility with Other
Curricula}\label{compatibility-with-other-curricula-3}
\addcontentsline{toc}{section}{Compatibility with Other Curricula}

\markright{Compatibility with Other Curricula}

JIMES (JIMES-AM-Ecosys)

\section*{Proportion of Final Grade}\label{proportion-of-final-grade-3}
\addcontentsline{toc}{section}{Proportion of Final Grade}

\markright{Proportion of Final Grade}

5\%

\section*{Module Coordinator}\label{module-coordinator-3}
\addcontentsline{toc}{section}{Module Coordinator}

\markright{Module Coordinator}

Prof.~Dr.~Christina Bogner (Institute of Geography)

\section*{Further Information}\label{further-information-3}
\addcontentsline{toc}{section}{Further Information}

\markright{Further Information}

There are no restrictions on the number of places available for IMES
students and BVDU students.

\chapter*{Internship}\label{internship}
\addcontentsline{toc}{chapter}{Internship}

\markboth{Internship}{Internship}

\begin{longtable}[]{@{}ll@{}}
\toprule\noalign{}
\endhead
\bottomrule\noalign{}
\endlastfoot
\textbf{Type of Module} & Specialisation Module \\
\textbf{Module Code} & IMES-SM-Int \\
\textbf{Workload} & 360 h \\
\textbf{ECTS} & 12 \\
\textbf{Term} & Semester 3 \\
\textbf{Offered Every} & SuSe \\
\textbf{Start} & SuSe \\
\textbf{Duration} & Minimum 8 weeks \\
\end{longtable}

\begin{longtable}[]{@{}lll@{}}
\toprule\noalign{}
Course Types & Contact Time & Self-study Time \\
\midrule\noalign{}
\endhead
\bottomrule\noalign{}
\endlastfoot
a) Internship & 320 h & 7.5 h \\
b) Seminar & 0.5 SWS & 32.5 h \\
\end{longtable}

\section*{Module Description}\label{module-description-4}
\addcontentsline{toc}{section}{Module Description}

\markright{Module Description}

The internship of two months duration is scheduled as part of the third
semester. It is an opportunity for students to develop research skills
and/or gain professional expertise as well as academic knowledge.
Students are expected to begin planning their internship early in the
semester to ensure a timely placement and registration. While the timing
of the internship may vary depending on individual arrangements, it
should be aligned with the study plan and registered with the IMES
Office before commencement. Students are encouraged to find their own
internship placements and make initial contact with prospective
organizations, as this builds communication and organizational skills.
The IMES Office can provide guidance in certain cases and share a
database of former internship hosts but does not guarantee placement.
The module includes a seminar in which students present and reflect on
their internship experiences. This setting provides a structured
opportunity for peer-to-peer learning, constructive feedback, and
academic exchange. Students are expected not only to defend their own
internship work but also to engage actively with the presentations of
their peers by asking questions, giving feedback, and discussing diverse
professional contexts relevant to environmental sciences.

\section*{Module Objectives}\label{module-objectives-3}
\addcontentsline{toc}{section}{Module Objectives}

\markright{Module Objectives}

\begin{itemize}
\tightlist
\item
  Facilitate the application of theoretical knowledge in real-world
  contexts through hands-on experience in professional settings.
\item
  Enhance problem-solving, analytical, and decision-making skills by
  engaging in practical projects related to the core domains of the
  program.
\item
  Cultivate professional competencies, including teamwork,
  communication, and leadership, through structured mentorship and
  collaboration with industry or research organizations.
\end{itemize}

\section*{Module Outcomes}\label{module-outcomes-3}
\addcontentsline{toc}{section}{Module Outcomes}

\markright{Module Outcomes}

At the end of this module, students will be able to:

\begin{itemize}
\tightlist
\item
  Demonstrate the ability to apply theoretical knowledge to practical
  challenges in professional or research environments.
\item
  Develop and execute project-based tasks with minimal supervision,
  showcasing initiative and problem-solving skills.
\item
  Communicate effectively through written reports, presentations, and
  discussions, tailored to both academic and professional audiences.
\item
  Demonstrate enhanced soft skills such as teamwork, adaptability, and
  professionalism in diverse settings.
\item
  Critically reflect on their internship experience to identify
  strengths, areas for improvement, and future career directions.
\item
  Create meaningful contributions to their host organization, supported
  by innovative ideas or solutions derived during the internship.
\end{itemize}

\section*{Module Content}\label{module-content-4}
\addcontentsline{toc}{section}{Module Content}

\markright{Module Content}

An internship matching the thematic focus of Environmental Sciences as
defined by the IMES program.

\textbf{Registration and documentation requirements}

Before starting the internship, students must register it with the IMES
Office. Registration requires:

\begin{itemize}
\tightlist
\item
  A brief written outline (1 page) describing the host organization, the
  planned tasks and responsibilities, and how the internship aligns with
  the thematic focus of Environmental Sciences as defined by the IMES
  program.
\item
  A confirmation from the host organization (in any reasonable format,
  such as an email or signed letter) indicating the planned duration and
  general scope of the internship. A formal contract is not mandatory,
  but may be required by the host organization.
\end{itemize}

After completing the internship, students must submit a certificate of
completion from the host organization. This document should briefly
state the internship duration and confirm that the agreed-upon tasks
were carried out.

\section*{Teaching Methods}\label{teaching-methods-4}
\addcontentsline{toc}{section}{Teaching Methods}

\markright{Teaching Methods}

\begin{enumerate}
\def\labelenumi{\alph{enumi})}
\tightlist
\item
  Internship\\
\item
  Seminar at UoC: student presentations, peer feedback, group discussion
\end{enumerate}

\section*{Prerequisites}\label{prerequisites-4}
\addcontentsline{toc}{section}{Prerequisites}

\markright{Prerequisites}

The student has acquired at least 30 ECTS from the first three
semesters.

\section*{Type of Examination}\label{type-of-examination-4}
\addcontentsline{toc}{section}{Type of Examination}

\markright{Type of Examination}

The supervisor will grade the internship based on an internship report
of 20 pages, written in English, which introduces institutional/NGO
setup, people encountered and worked alongside, including their tasks
and positions; activities undertaken, including the content, aim, and
timeframe of the work; policy areas touched upon; skills acquired;
insight into institutional practice obtained. The students are required
to present and defend their report during the seminar.

\textbf{1. Internship report (70\%)}

The internship report, written in English (20 pages), should contain the
following:

\begin{itemize}
\tightlist
\item
  Introduction to the institutional/NGO setup (10\%): Describing the
  organization, objectives, and operations comprehensively.
\item
  Tasks and people encountered (10\%): Detailed account of roles,
  responsibilities, and interactions with colleagues.
\item
  Activities undertaken (10\%): Explanation of content, objectives,
  timelines, the scope of the work and outputs.
\item
  Policy areas and skills acquired (10\%): Identification of relevant
  policy domains and skills developed.
\item
  Insights into institutional practices and self-reflection (30\%):
  In-depth analysis of institutional processes, key learnings,
  challenges faced, and personal growth.
\end{itemize}

\textbf{2. Presentation and defence of the report (30\%)}

The student presents their report to a panel of IMES staff (20 minutes),
assessed on:

\begin{itemize}
\tightlist
\item
  Clarity, organization, and coherence of the presentation (15\%).
\item
  Ability to answer questions and demonstrate understanding of the
  internship's context, activities, and outcomes (15\%).
\end{itemize}

\section*{Credits Awarded}\label{credits-awarded-4}
\addcontentsline{toc}{section}{Credits Awarded}

\markright{Credits Awarded}

Credit points are awarded upon regular and active participation in the
internship and the seminar, as well as successful completion of the
module's examinations with a minimum passing grade of 4.0 (sufficient).
A certificate of completion from the host organization is normally
required, but alternative proof may be accepted in justified cases.

The internship must be registered with the IMES Office before
commencement. Registration requires a written outline of the planned
tasks and a confirmation from the host organization.

\section*{Compatibility with Other
Curricula}\label{compatibility-with-other-curricula-4}
\addcontentsline{toc}{section}{Compatibility with Other Curricula}

\markright{Compatibility with Other Curricula}

None

\section*{Proportion of Final Grade}\label{proportion-of-final-grade-4}
\addcontentsline{toc}{section}{Proportion of Final Grade}

\markright{Proportion of Final Grade}

0\%. The internship is graded but does not contribute to the final
grade.

\section*{Module Coordinator}\label{module-coordinator-4}
\addcontentsline{toc}{section}{Module Coordinator}

\markright{Module Coordinator}

Dr.~Hannes Laermanns (Institute of Geography) -- IMES Office

\section*{Further Information}\label{further-information-4}
\addcontentsline{toc}{section}{Further Information}

\markright{Further Information}

No restriction on number of places for IMES students.

\chapter*{Master Thesis and
Colloquium}\label{master-thesis-and-colloquium}
\addcontentsline{toc}{chapter}{Master Thesis and Colloquium}

\markboth{Master Thesis and Colloquium}{Master Thesis and Colloquium}

\begin{longtable}[]{@{}ll@{}}
\toprule\noalign{}
\endhead
\bottomrule\noalign{}
\endlastfoot
\textbf{Type of Module} & Master thesis \\
\textbf{Module Code} & IMES-MasterThesis \\
\textbf{Workload} & 900 h \\
\textbf{ECTS} & 30 \\
\textbf{Term} & Semester 4 \\
\textbf{Offered Every} & WiSe / SuSe \\
\textbf{Start} & WiSe / SuSe \\
\textbf{Duration} & 1 semester \\
\end{longtable}

\begin{longtable}[]{@{}lll@{}}
\toprule\noalign{}
Course Types & Contact Time & Self-study Time \\
\midrule\noalign{}
\endhead
\bottomrule\noalign{}
\endlastfoot
a) Master Thesis & 875 h & 20 h \\
b) Colloquium & 4 h & 1 h \\
\end{longtable}

\section*{Module Objectives and
Outcomes}\label{module-objectives-and-outcomes-1}
\addcontentsline{toc}{section}{Module Objectives and Outcomes}

\markright{Module Objectives and Outcomes}

\subsection*{a) Master Thesis}\label{a-master-thesis}
\addcontentsline{toc}{subsection}{a) Master Thesis}

The Master's thesis is an independent academic examination designed to
demonstrate the candidate's ability to address a well-defined problem
within the field of study using appropriate scientific methods within a
specified timeframe. The thesis may be completed in any subject of the
IMES program and must be written in English.

Throughout the thesis phase, students receive individual academic
supervision. Supervisors provide feedback on the exposé, guidance during
the research and writing process, and support for the mid-term
presentation. This structure is designed to help students manage their
time effectively and produce work of high academic quality.

At the beginning of the thesis process, students are required to submit
an exposé (approximately 2--4 pages), outlining the research question,
objectives, theoretical background, methodology, and expected structure
of the thesis. The exposé must demonstrate the feasibility and academic
relevance of the project and serves as the basis for discussion and
feedback with the supervisor.

Approximately halfway through the working period, a mid-term
presentation is mandatory. In this presentation, the student presents
their progress and discusses preliminary results in a seminar with their
supervisors and peers. Both the exposé and the mid-term presentation are
ungraded coursework (``Studienleistungen'') and must be passed.

The maximum timeframe for completing the Master's thesis is six months
from the date the topic is assigned. The thesis should not exceed 100 A4
pages.

Further details on the Master's thesis are provided in the examination
regulations (§21).

\subsection*{b) Colloquium (Oral defense of the master's
thesis)}\label{b-colloquium-oral-defense-of-the-masters-thesis}
\addcontentsline{toc}{subsection}{b) Colloquium (Oral defense of the
master's thesis)}

The colloquium is conducted as an interdisciplinary individual
examination. The candidate presents the contents of their Master's
thesis in a 10- to 15-minute presentation, followed by questions and
discussion with the examiners. The colloquium lasts between 40 and 60
minutes. Students and staff of the IMES program are permitted to attend
the colloquium unless the candidate objects. Participation of observers
does not include the deliberation or announcement of results.

Further details on the Colloquium are given in the examination
regulations (§21).

\section*{Module Content}\label{module-content-5}
\addcontentsline{toc}{section}{Module Content}

\markright{Module Content}

The module consists of four parts:\\
(1) submission of an exposé outlining the research question,
methodology, and structure;\\
(2) a mandatory mid-term presentation to report on progress;\\
(3) completion of an independent written thesis;\\
(4) a final oral colloquium to defend the thesis.\\
The thesis should not exceed 100 pages and must be completed within six
months of topic assignment.

\section*{Teaching Methods}\label{teaching-methods-5}
\addcontentsline{toc}{section}{Teaching Methods}

\markright{Teaching Methods}

Individual supervision including support during the exposé phase,
feedback after the mid-term presentation, and academic consultations
throughout the research and writing process.

\section*{Prerequisites}\label{prerequisites-5}
\addcontentsline{toc}{section}{Prerequisites}

\markright{Prerequisites}

The student has acquired at least 60 ECTS from the first three semesters
and successfully completed all mandatory modules.

\section*{Type of Examination}\label{type-of-examination-5}
\addcontentsline{toc}{section}{Type of Examination}

\markright{Type of Examination}

\begin{itemize}
\tightlist
\item
  Written Master thesis (weight: 75\%)\\
\item
  Oral defense (weight: 25\%)\\
\item
  Exposé (ungraded course work ``Studienleistung'')\\
\item
  Mid-term presentation (ungraded course work ``Studienleistung'')
\end{itemize}

\section*{Credits Awarded}\label{credits-awarded-5}
\addcontentsline{toc}{section}{Credits Awarded}

\markright{Credits Awarded}

Credit points are awarded when the ungraded exposé and mid-term
presentation were successfully passed and the examinations of the module
parts have been successfully completed with the minimum grade 4.0
(sufficient).

\section*{Compatibility with Other
Curricula}\label{compatibility-with-other-curricula-5}
\addcontentsline{toc}{section}{Compatibility with Other Curricula}

\markright{Compatibility with Other Curricula}

None

\section*{Proportion of Final Grade}\label{proportion-of-final-grade-5}
\addcontentsline{toc}{section}{Proportion of Final Grade}

\markright{Proportion of Final Grade}

25\%

\section*{Module Coordinator}\label{module-coordinator-5}
\addcontentsline{toc}{section}{Module Coordinator}

\markright{Module Coordinator}

Head of the IMES Examination Committee

\section*{Further Information}\label{further-information-5}
\addcontentsline{toc}{section}{Further Information}

\markright{Further Information}

None

\part{EXAMPLE SCHEDULES}

\chapter*{Schedules}\label{schedules}
\addcontentsline{toc}{chapter}{Schedules}

\markboth{Schedules}{Schedules}


\backmatter


\end{document}
